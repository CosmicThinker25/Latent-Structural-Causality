\documentclass[11pt,a4paper]{article}

\usepackage{amsmath,amssymb}
\usepackage{geometry}
\usepackage{setspace}
\usepackage{hyperref}
\usepackage{authblk}

\geometry{margin=1in}
\onehalfspacing

\title{\textbf{Latent Structural Causality:\\
Defining Physical Existence in Pre-Observable Regimes}}

\author[1]{Cosmic Thinker}
\affil[1]{Independent Researcher}
\date{}

\begin{document}
\maketitle

\begin{abstract}
Standard cosmological models often postulate physical entities—such as gravity, time, or radiation—in regimes where they cannot produce observable effects due to extreme coupling, symmetry, or phase coherence. This raises a fundamental epistemological problem: in what sense can an entity be said to exist before it is observable? In this paper, we propose the concept of \emph{Latent Structural Causality} (LSC) to describe physical structures that constrain the state space of a system without producing operative dynamics within a given regime.

Drawing on established examples from the Standard Model (the latency of mass prior to electroweak symmetry breaking) and cosmology (the latency of photons prior to recombination), we formalize the conditions under which a physical interaction is structurally present yet operationally silent. We then apply this framework to phase-driven cosmological scenarios, including CPT-symmetric models, to show how gravity and time can emerge from a pre-observable phase without invoking creation \emph{ex nihilo}. We conclude that physical existence in early-universe cosmology should be defined not by immediate observability, but by \emph{Retrospective Necessity}: the requirement that the present coherent universe demands specific latent structural constraints in its past.
\end{abstract}

\vspace{1em}
\noindent\textbf{Keywords:} Latent Structural Causality; Foundations of Cosmology; CPT Symmetry; Ontology of Physics; Early Universe.

\section{Introduction}

Modern cosmology frequently confronts regimes in which familiar physical concepts—gravity, light, mass, or even time—are formally defined within a theory but appear incapable of producing observable effects. This tension is particularly acute when discussing the early universe, pre-recombination epochs, or models that seek to avoid an initial singularity. The standard empiricist criterion of existence, equating reality with direct observability, becomes inadequate in such contexts.

This paper proposes a precise conceptual framework for addressing this issue. By introducing the notion of \emph{Latent Structural Causality}, we distinguish between the structural presence of a physical interaction in the fundamental formalism and its operative manifestation within a given regime. This distinction allows cosmology to discuss pre-observable phases without resorting to metaphysical speculation.

\section{Defining Latent Structural Causality (LSC)}

Standard physical epistemology relies heavily on observability as a criterion for physical existence. In most laboratory contexts, an entity is considered real insofar as it produces measurable effects on a detector or induces a non-vanishing contribution to the energy--momentum tensor. Cosmology, however, confronts us with regimes—such as the pre-recombination universe or epochs prior to symmetry breaking—where certain fundamental interactions are formally present in the theory but dynamically unable to generate distinguishable observables.

To address this problem, we introduce the concept of \emph{Latent Structural Causality} (LSC), which provides a rigorous distinction between structural presence and operative manifestation.

\subsection{Formal Definition}

Consider a physical system described by a Hamiltonian $\mathcal{H}$ acting on a Hilbert space $\mathcal{H}_{\text{sys}}$, and let $|\Psi(t)\rangle$ denote the cosmological state. An interaction represented by an operator $\mathcal{O}$ possesses \emph{Latent Structural Causality} within a regime $t < t_c$ if:

\begin{enumerate}
\item \textbf{Structural Presence.} The operator $\mathcal{O}$ appears explicitly in the fundamental Hamiltonian or as a generator of symmetries:
\begin{equation}
\mathcal{H} = \mathcal{H}_0 + \lambda \mathcal{O}.
\end{equation}

\item \textbf{Null Operativity (Silence Condition).}
\begin{equation}
\langle \Psi(t) | \mathcal{O} | \Psi(t) \rangle \approx 0,
\qquad
\Delta \mathcal{O} \approx 0,
\end{equation}
such that the interaction produces no work, no information flow, and no distinguishable historical record.

\item \textbf{Conditional Operativity.} There exists a critical parameter $\chi$ for which
\begin{equation}
\langle \Psi(t>t_c) | \mathcal{O} | \Psi(t>t_c) \rangle \neq 0,
\end{equation}
rendering the interaction dynamically operative.
\end{enumerate}

\subsection{Distinction from Potentiality}

Latent Structural Causality must be distinguished from philosophical notions of mere potentiality. A latent interaction is not something that might exist under different circumstances; it is a structural constraint required for theoretical consistency. The electroweak epoch provides a canonical example: the Higgs field and Yukawa couplings are present before symmetry breaking, yet mass remains dynamically silent until the transition occurs.

\subsection{The Principle of Retrospective Necessity}

Because latent interactions generate no observables within their own regime, their existence cannot be established by direct measurement. Instead, they are inferred via \emph{Retrospective Necessity}: the requirement that the observable universe at $t>t_c$ would be mathematically incoherent if the latent structure had not been present earlier.

\section{Case Studies: Latent Causality in Established Physics}

We now examine canonical examples where LSC is already implicit in standard physics.

\subsection{Radiation Before Recombination}

Photons exist from the earliest hot phases of the universe, yet prior to recombination they are tightly coupled to the primordial plasma. Radiation is structurally present but unable to propagate freely or transmit information. Recombination does not create photons; it activates their causal efficacy, producing the cosmic microwave background as a relic of a previously latent structure.

\subsection{Mass Before Electroweak Symmetry Breaking}

Before electroweak symmetry breaking, particles behave as massless despite the structural presence of Higgs and Yukawa terms. Mass is latent, becoming operative only after the symmetry-breaking transition. Observable inertia retroactively justifies the earlier structural presence of mass.

\subsection{Gravity Prior to Classical Structure Formation}

In a perfectly homogeneous and isotropic early universe, gravity is structurally encoded in the field equations but produces no measurable effects. Without gradients or separable systems, curvature remains operationally silent. The emergence of structure marks the transition from latent to operative gravity.

\section{Latent Causality in Phase-Driven Cosmologies}

Phase-driven cosmologies provide a natural setting for the application of LSC.

\subsection{Phase Synchronization as Latency}

In CPT-symmetric ``siamese'' models, the early universe is characterized by maximal phase coherence between sectors. Interactions dependent on gradients—most notably gravity—are structurally present but dynamically silent, satisfying the defining conditions of LSC.

\subsection{Desynchronization as Conditional Operativity}

A minimal phase desynchronization acts as the critical parameter $\chi$. This transition is not an act of creation, but a regime change: gradients become admissible, rendering gravity and time operative. LSC thus legitimizes the study of pre-geometric phases as physically real and structurally necessary.

\section{Discussion: The Epistemology of Retrospective Necessity}

LSC challenges strict empiricism without abandoning scientific rigor. In early-universe cosmology, existence is inferred not from direct observation, but from structural coherence. The footprint does not cause the foot, yet necessitates its prior existence. To avoid metaphysical inflation, LSC enforces continuity, economy, and falsifiability via observable relics.

Physical existence is therefore regime-dependent:
\begin{itemize}
\item \textbf{Structural existence}: inclusion in the fundamental formalism.
\item \textbf{Operative existence}: capacity to generate work, information, or observable history.
\end{itemize}

Observables emerge from non-operativity, not from non-existence.

\section{Conclusion: Toward a Regime-Dependent Ontology}

Latent Structural Causality provides a coherent framework for discussing pre-observable regimes in cosmology. By distinguishing structural presence from operativity, it replaces singular creation narratives with transitions of regime. Retrospective Necessity emerges as the appropriate epistemological standard for foundational cosmology, legitimizing scientific inquiry beyond direct observation.

\begin{thebibliography}{99}

\bibitem{Weinberg1967}
S. Weinberg, ``A Model of Leptons,'' \emph{Phys. Rev. Lett.} \textbf{19}, 1264 (1967).

\bibitem{Peebles1968}
P. J. E. Peebles, ``Recombination of the Primeval Plasma,'' \emph{Astrophys. J.} \textbf{153}, 1 (1968).

\bibitem{Boyle2018}
L. Boyle, K. Finn, and N. Turok, ``CPT-Symmetric Universe,'' \emph{Phys. Rev. Lett.} \textbf{121}, 251301 (2018).

\bibitem{Bell1987}
J. S. Bell, \emph{Speakable and Unspeakable in Quantum Mechanics}, Cambridge University Press (1987).

\bibitem{Wheeler1990}
J. A. Wheeler, ``Information, physics, quantum: The search for links,'' in
\emph{Complexity, Entropy, and the Physics of Information}, Addison--Wesley (1990).

\bibitem{Barbour1999}
J. Barbour, \emph{The End of Time}, Oxford University Press (1999).

\end{thebibliography}

\end{document}

\documentclass[11pt,a4paper]{article}

\usepackage[spanish]{babel}
\usepackage[utf8]{inputenc}
\usepackage[T1]{fontenc}

\usepackage{amsmath,amssymb}
\usepackage{geometry}
\usepackage{setspace}
\usepackage{hyperref}
\usepackage{authblk}

\geometry{margin=1in}
\onehalfspacing

\title{\textbf{Causalidad Estructural Latente:\\
Definiendo la Existencia Física en Regímenes Preobservables}}

\author[1]{Cosmic Thinker}
\affil[1]{Investigador Independiente}
\date{}

\begin{document}
\maketitle

\begin{abstract}
Los modelos cosmológicos estándar suelen postular entidades físicas —como la gravedad, el tiempo o la radiación— en regímenes donde no pueden producir efectos observables debido a acoplamientos extremos, simetrías o coherencia de fase. Esto plantea un problema epistemológico fundamental: ¿en qué sentido puede decirse que una entidad existe antes de ser observable? En este trabajo proponemos el concepto de \emph{Causalidad Estructural Latente} (CEL) para describir estructuras físicas que constriñen el espacio de estados de un sistema sin producir dinámicas operativas dentro de un régimen dado.

Basándonos en ejemplos consolidados del Modelo Estándar (la latencia de la masa antes de la ruptura electrodébil) y de la cosmología (la latencia de los fotones antes de la recombinación), formalizamos las condiciones bajo las cuales una interacción física está estructuralmente presente pero permanece operacionalmente silenciosa. Aplicamos este marco a escenarios cosmológicos impulsados por transiciones de fase, incluidos modelos CPT-simétricos, mostrando cómo la gravedad y el tiempo pueden emerger de una fase preobservable sin invocar creación \emph{ex nihilo}. Concluimos que la existencia física en la cosmología del universo temprano no debe definirse por la observabilidad inmediata, sino por la \emph{Necesidad Retrospectiva}: el requisito de que el universo coherente actual demande la existencia de restricciones estructurales latentes en su pasado.
\end{abstract}

\vspace{1em}
\noindent\textbf{Palabras clave:} Causalidad Latente; Fundamentos de la Cosmología; Simetría CPT; Ontología de la Física; Universo Temprano.

\section{Introducción}

La cosmología moderna se enfrenta con frecuencia a regímenes en los que conceptos físicos familiares —gravedad, luz, masa o incluso tiempo— están formalmente definidos dentro de una teoría, pero parecen incapaces de producir efectos observables. Esta tensión es especialmente aguda al discutir el universo temprano, las épocas previas a la recombinación o los modelos que buscan evitar una singularidad inicial. El criterio empirista estándar de existencia, que equipara la realidad con la observabilidad directa, resulta insuficiente en estos contextos.

Este trabajo propone un marco conceptual preciso para abordar este problema. Introduciendo la noción de \emph{Causalidad Estructural Latente}, distinguimos entre la presencia estructural de una interacción física en el formalismo fundamental y su manifestación operativa dentro de un régimen determinado. Esta distinción permite a la cosmología discutir fases preobservables sin recurrir a la especulación metafísica.

\section{Definición de la Causalidad Estructural Latente (CEL)}

La epistemología física estándar se apoya fuertemente en la observabilidad como criterio de existencia física. En la mayoría de los contextos experimentales, una entidad se considera real en la medida en que produce efectos medibles sobre un detector o induce una contribución no nula al tensor energía–momento. Sin embargo, la cosmología nos enfrenta a regímenes —como el universo previo a la recombinación o las épocas anteriores a la ruptura de simetrías— en los que ciertas interacciones fundamentales están formalmente presentes en la teoría, pero son dinámicamente incapaces de generar observables distinguibles.

Para abordar este problema, introducimos el concepto de \emph{Causalidad Estructural Latente} (CEL), que proporciona una distinción rigurosa entre presencia estructural y manifestación operativa.

\subsection{Definición Formal}

Consideremos un sistema físico descrito por un Hamiltoniano $\mathcal{H}$ que actúa sobre un espacio de Hilbert $\mathcal{H}_{\text{sys}}$, y sea $|\Psi(t)\rangle$ el estado cosmológico del sistema. Una interacción representada por un operador $\mathcal{O}$ posee \emph{Causalidad Estructural Latente} dentro de un régimen $t < t_c$ si se cumplen las siguientes condiciones:

\begin{enumerate}
\item \textbf{Presencia Estructural.} El operador $\mathcal{O}$ aparece explícitamente en el Hamiltoniano fundamental o como generador de simetrías:
\begin{equation}
\mathcal{H} = \mathcal{H}_0 + \lambda \mathcal{O}.
\end{equation}

\item \textbf{Operatividad Nula (Condición de Silencio).}
\begin{equation}
\langle \Psi(t) | \mathcal{O} | \Psi(t) \rangle \approx 0,
\qquad
\Delta \mathcal{O} \approx 0,
\end{equation}
de modo que la interacción no produce trabajo, flujo de información ni registro histórico distinguible.

\item \textbf{Operatividad Condicional.} Existe un parámetro crítico $\chi$ tal que
\begin{equation}
\langle \Psi(t>t_c) | \mathcal{O} | \Psi(t>t_c) \rangle \neq 0,
\end{equation}
haciendo que la interacción se vuelva dinámicamente operativa.
\end{enumerate}

\subsection{Distinción frente a la Potencialidad}

La Causalidad Estructural Latente debe distinguirse de nociones filosóficas de mera potencialidad. Una interacción latente no es algo que podría existir bajo otras circunstancias; es una restricción estructural necesaria para la coherencia teórica. La época electrodébil proporciona un ejemplo canónico: el campo de Higgs y los acoplamientos de Yukawa están presentes antes de la ruptura de simetría, pero la masa permanece dinámicamente silenciosa hasta que ocurre la transición.

\subsection{El Principio de Necesidad Retrospectiva}

Dado que las interacciones latentes no generan observables dentro de su propio régimen, su existencia no puede establecerse mediante mediciones directas. En su lugar, se infieren mediante la \emph{Necesidad Retrospectiva}: el requisito de que el universo observable en $t>t_c$ sería matemáticamente incoherente si la estructura latente no hubiera estado presente anteriormente.

\section{Estudios de Caso: Causalidad Latente en la Física Establecida}

Examinamos ahora ejemplos canónicos en los que la CEL ya está implícita en la física estándar.

\subsection{Radiación antes de la Recombinación}

Los fotones existen desde las fases más tempranas y calientes del universo, pero antes de la recombinación están fuertemente acoplados al plasma primordial. La radiación está estructuralmente presente, pero no puede propagarse libremente ni transmitir información. La recombinación no crea fotones; activa su eficacia causal, produciendo el fondo cósmico de microondas como vestigio de una estructura previamente latente.

\subsection{Masa antes de la Ruptura Electrodébil}

Antes de la ruptura de simetría electrodébil, las partículas se comportan como masivas nulas a pesar de la presencia estructural del Higgs y los términos de Yukawa. La masa es latente y se vuelve operativa solo tras la transición. La inercia observable justifica retrospectivamente la presencia estructural previa de la masa.

\subsection{Gravedad antes de la Formación de Estructura Clásica}

En un universo temprano perfectamente homogéneo e isotrópico, la gravedad está estructuralmente codificada en las ecuaciones de campo, pero no produce efectos medibles. Sin gradientes ni subsistemas separables, la curvatura permanece operacionalmente silenciosa. La formación de estructura marca la transición de la gravedad latente a la operativa.

\section{Causalidad Latente en Cosmologías Impulsadas por Fase}

Las cosmologías impulsadas por transiciones de fase proporcionan un marco natural para la aplicación de la CEL.

\subsection{Sincronización de Fase como Latencia}

En modelos CPT-simétricos ``siameses'', el universo temprano se caracteriza por una coherencia de fase máxima entre sectores. Las interacciones dependientes de gradientes —en particular la gravedad— están estructuralmente presentes pero dinámicamente silenciadas, cumpliendo exactamente la definición de CEL.

\subsection{Desincronización como Operatividad Condicional}

Una desincronización mínima de fase actúa como el parámetro crítico $\chi$. Esta transición no es un acto de creación, sino un cambio de régimen: los gradientes se vuelven admisibles y la gravedad y el tiempo pasan a ser operativos. La CEL legitima así el estudio de fases pregeométricas como regímenes físicamente reales y estructuralmente necesarios.

\section{Discusión: La Epistemología de la Necesidad Retrospectiva}

La CEL desafía el empirismo estricto sin abandonar el rigor científico. En la cosmología del universo temprano, la existencia no se infiere por observación directa, sino por coherencia estructural. La huella no causa el pie, pero exige su existencia previa. Para evitar la inflación metafísica, la CEL impone continuidad, economía y falsabilidad a través de vestigios observables.

La existencia física es, por tanto, dependiente del régimen:
\begin{itemize}
\item \textbf{Existencia estructural}: inclusión en el formalismo fundamental.
\item \textbf{Existencia operativa}: capacidad de generar trabajo, información o historia observable.
\end{itemize}

Los observables emergen de la no operatividad, no de la no existencia.

\section{Conclusión: Hacia una Ontología Dependiente del Régimen}

La Causalidad Estructural Latente proporciona un marco coherente para discutir regímenes preobservables en cosmología. Al distinguir entre presencia estructural y operatividad, reemplaza narrativas de creación singular por transiciones de régimen. La Necesidad Retrospectiva emerge así como el estándar epistemológico adecuado para la cosmología fundamental, legitimando la exploración científica más allá de la observación directa.

\begin{thebibliography}{99}

\bibitem{Weinberg1967}
S. Weinberg, ``A Model of Leptons,'' \emph{Phys. Rev. Lett.} \textbf{19}, 1264 (1967).

\bibitem{Peebles1968}
P. J. E. Peebles, ``Recombination of the Primeval Plasma,'' \emph{Astrophys. J.} \textbf{153}, 1 (1968).

\bibitem{Boyle2018}
L. Boyle, K. Finn, and N. Turok, ``CPT-Symmetric Universe,'' \emph{Phys. Rev. Lett.} \textbf{121}, 251301 (2018).

\bibitem{Bell1987}
J. S. Bell, \emph{Speakable and Unspeakable in Quantum Mechanics}, Cambridge University Press (1987).

\bibitem{Wheeler1990}
J. A. Wheeler, ``Information, physics, quantum: The search for links,'' in
\emph{Complexity, Entropy, and the Physics of Information}, Addison--Wesley (1990).

\bibitem{Barbour1999}
J. Barbour, \emph{The End of Time}, Oxford University Press (1999).

\end{thebibliography}

\end{document}
